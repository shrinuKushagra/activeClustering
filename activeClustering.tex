\documentclass[orivec]{llncs}
\usepackage{llncsdoc}

\def\COMPLETE{}
\usepackage[boxruled]{algorithm2e}
\usepackage{amsmath,amssymb,amstext}
\usepackage[margin=1in]{geometry}

\usepackage{graphicx}
\usepackage{hyperref}
\usepackage[toc,page]{appendix}

\usepackage{color}
\usepackage[toc,page]{appendix}
\usepackage{xspace}
\usepackage{enumitem}
\usepackage{times}
\usepackage{float}
\usepackage{capt-of}

\usepackage{tikz}
\usetikzlibrary{shapes, calc, arrows, through, intersections, decorations.pathreplacing, patterns}

\newcommand{\mc}{\mathcal}
\newcommand{\mb}{\mathbf}
\DeclareMathOperator*{\argmin}{arg\,min}
\DeclareMathOperator*{\argmax}{arg\,max}
\DeclareMathOperator{\vcdim}{VC-Dim}
\DeclareMathOperator{\vol}{vol}

\renewcommand{\qed}{\hfill\ensuremath{\blacksquare}}
\renewcommand\labelitemi{$\bullet$}

\makeatletter  %% this is crucial
\renewcommand\subsubsection{\@startsection{subsubsection}{3}{\z@}%
   {-18\p@ \@plus -4\p@ \@minus -4\p@}%
   {8\p@ \@plus 4\p@ \@minus 4\p@}%     <-- this
   {\normalfont\normalsize\bfseries\boldmath
   \rightskip=\z@ \@plus 8em \pretolerance=10000}}
\makeatother   %% this is crucial
\setcounter{secnumdepth}{3}


\newcommand{\todo}{\textcolor{blue}{[TODO]}\xspace}
\newcommand{\complete}{\textcolor{red}{[TO BE COMPLETED]}\xspace}
\newcommand{\wip}{\textcolor{red}{[Work in progress]}\xspace}
\newcommand{\multlinecomment}[1]{\directlua{-- #1}}




\newcommand{\fix}{\textcolor{blue}{[FIX]}\xspace}
\newcommand{\q}{\textcolor{blue}{[?]}\xspace}

\title{Active clustering}
\author{Student submission}
%\institute{School of Computer Science\\University of Waterloo\\ Waterloo, ON, N2L 3G1 \\CANADA \\ \email{\{skushagr@,shai@cs.\}uwaterloo.ca}}


%%%%% Document Body %%%%%
\begin{document}
\maketitle

\begin{abstract}
We should write a good abstract here.
\end{abstract}

\section{Introduction}
We should introduce stuff here.

\subsection{Related work}
We should discuss related work here.


\section{Problem formulation}

\subsection{Center-based clustering}
\begin{itemize}[nolistsep]
\item Center-of-mass + EU + Steiner (our focus is natural).
\item complexity of algorithm (run-time polynomial in $n$ and $k$).
\end{itemize}

\subsection{Clustering with query}
\begin{itemize}[nolistsep]
\item Complexity (membership) query
\item Algorithm solves it with query complexity $q(n, k)$ and time complexity $t(n, k)$.
\item Other forms of query (in Appendix \ref{appendix:diffQueryModels})
\end{itemize}

\subsection{Clustering under niceness assumption}
\begin{definition}[$\gamma$-margin]
\label{defn:alphacp}
Given a clustering instance $(\mc X, d)$ from some metric space $M$ and the number of clusters $k$. We say that a clustering $\mc C_{\mc X} = \{C_1, \ldots, C_k\}$ induced by centers $\mu_1, \ldots, \mu_k \in M$ has $\gamma$-margin w.r.t $\mc X$ and $k$ if the following holds. For all $x \in C_i$ and $y \in C_j$, 
$$\gamma d(x, c_i) < d(y, c_i)$$
\end{definition}

\noindent We will assume that the target clustering has $\gamma$-margin. In section \ref{section:clusteringWithQuery}, we will give algorithms (using queries) which recover the target clustering when $\gamma > 1$. In section \ref{section:lowerBounds}, we give lower bounds under $\gamma$-margin assumption.  

In literature, similar assumptions have been considered before. One of them is $\alpha$-center proximity \cite{citeStuff}. Comparison of our notion of margin against center proximity is orthogonal to the problem being addressed in this paper. Hence, all these details have been moved to appendix \ref{appendix:gammaMrginVsAlphaCenter}. However, an important point to note from the discussion in the appendix is that the results (both upper and lower bounds) that have been obtained under $\alpha$-center proximity can also be obtained under $\gamma$-margin for `reasonably small' values of $\gamma$. 

\section{Clustering with queries}
\label{section:clusteringWithQuery}

We will assume that the target clustering $\mc C_{\mc X} = \{C_1, \ldots, C_k\}$ induced by centers $\mu_1, \ldots, \mu_k$ has the following property. For all $i$, $\mu_i$ is the mean of all the points in cluster $C_i$, that is,
\begin{align}
\mu_i = \frac{1}{|C_i|}\sum_{x \in C_i} x \stepcounter{equation}\tag{P\theequation} \label{property:targetClust}
\end{align}
Note that the optimal solution to the euclidean $k$-means cost has this property. Our algorithm is more general and works for any target clustering which has the above property (\ref{property:targetClust}).

Intuitively, our algorithm (Alg. \ref{alg:steinerQueryPositive}) does the following. In the first phase, it randomly queries points from $\mc X$ till it has a sufficient number of points from one one cluster (say $C_p$). It uses the mean of these points to approximate the cluster center. In the second phase, the algorithm binary searches for a point $b_{idx}$ in $C_p$ which is at maximum distance from $\mu_p'$. Then, it clusters and deletes all points whose distance is $\le d(b_{idx}, \mu_p')$. We will show that this process guarantees that (as long as $\mu_p'$ is a good approximation of the actual center $\mu_p$) and $\gamma > 1$, all the points from $C_p$ are now recovered. We then repeat this process $k$ times to recover all the clusters.

The details of our approach is stated precisely in Alg. \ref{alg:steinerQueryPositive}. Thm. \ref{thm:steinerQueryPositive} shows that if $\gamma > 1$ then our algorithm recovers the target clustering with high probability. Next, we give bounds on the time and query complexity of our algorithm. Thm. \ref{thm:steinerQueryPositiveComplexity} shows that our approach needs $O(k\log n)$ queries and runs in $O(kn\log n)$ times. 

\RestyleAlgo{boxruled} 
\SetAlgoNoLine
\begin{algorithm}[h]
 \KwIn{Clustering instance $\mc X$, oracle $\mc O$, the number of clusters $k$ and parameters $\epsilon, \delta \in (0, 1)$}
 \KwOut{A clustering $\mc C$ of the set $\mc X$}

 \vspace{0.5em} $\mc C = \{\}$\\ 
 $\mc S_{1} = \mc X$\\
 $\eta = c\frac{\log k + \log(1/\delta)}{\epsilon^4}$\\
 \For{$i = 1$ to $k$}{
 	\vspace{0.7em}\textbf{Phase 1}\\
 	$l = k \eta + 1$\;
	Query the label of $l$ points (uniformly at random) from $\mc S_i$. Let $\mc A$ be the set of those points.\\
	For $1 \le t \le i$, let $A_t \subseteq \mc A$ be the set of points with label $i$. That is, \begin{center}$A_t = \{x \in \mc A : x \in C_t\}.$\end{center} 
	Choose any $A_p$ such that $|A_p| > \eta$.\\
	$\mu_p' := \frac{1}{|A_p|}\sum_{x \in A_p} x$.\\
	\vspace{1.5em}\textbf{Phase 2}\\
	Sort $x \in \mc S_i$ in increasing order of $d(x, \mu_p')$.\\
	Binary search over $\mc S_i$, to find an index $idx$ such that $b_{idx} \in C_p$ and $b_{idx+1} \not\in C_p$. (This step involves making queries to the oracle $\mc O$).\\ %We will later prove that $d(b_{idx}, c_p') = \max_{x \in C_p}d(x, c_p')$.
	$C_p' = \{x \in \mc S_i: d(x, \mu_p') \le d(b_{idx}, \mu_p')\}$.\\
	$S_{i+1} = S_{i}\setminus C_p'$.\\
	$\mc C = \mc C \cup C_p'$
 }
 Output $\mc C$.
 \label{alg:steinerQueryPositive}
 \caption{Algorithm for $\gamma(> 1)$-margin instances with queries}
\end{algorithm}


\begin{lemma}
\label{lemma:hasGammaMargin}
Given a clustering instance $(\mc X, d)$ in some metric space $M$. Let $\mc C_{\mc X} = \{C_1, \ldots, C_k\}$ be a clustering of $\mc X$ induced by centers $\mu_1, \ldots, \mu_k \in M$ which satisfies $\gamma$-margin and $\mu_i$ is the mean of points in $C_i$. Let $\mu_i' \in M$ such that $d(\mu_i, \mu_i') \le r(C_i)\epsilon$. If $\gamma \ge 1 + 2\epsilon$, then for all $x \in C_i$ and for all $y \in C_j$
$$d(x, \mu_i') < d(y, \mu_i')$$  
\end{lemma}

\begin{proof}
Fix any $x \in C_i$ and $y \in C_j$. $d(x, \mu_i') \le d(x, \mu_i)+d(\mu_i, \mu_i') \le r(C_i) (1+\epsilon)$. Similarly, $d(y, \mu_i') \ge d(y, \mu_i) - d(\mu_i, \mu_i') > (\gamma -\epsilon)r(C_i)$. Combining the two, we get that $d(x, \mu_i') < \frac{1+\epsilon}{\gamma-\epsilon}d(y, \mu_i')$. 
\qed
\end{proof}

\begin{lemma}
\label{lemma:phase1}
Let the framework be as in Lemma \ref{lemma:hasGammaMargin}. Let $A_p, C_p, \mu_p$ and $\mu_p', \eta$ be as defined is Alg. \ref{alg:steinerQueryPositive}. If $|A_p| > \eta$ then with probability atmost $\delta/k$, $d(\mu_p, \mu_p') > r(C_p)\epsilon$.
\end{lemma}
\begin{proof}
Define a uniform distribution $U$ over $C_p$. Then the mean of the distribution is $\mu_p$. Now, using Thm. \ref{thm:genHoeff} from Appendix \ref{appendixsection:conIneq} completes the proof.
\qed
\end{proof}

\begin{theorem}
\label{thm:steinerQueryPositive}
Given a clustering instance $(\mc X, d)$ in some metric space $M$. Let $\mc C_{\mc X} = \{C_1, \ldots, C_k\}$ be a clustering of $\mc X$ induced by centers $\mu_1, \ldots, \mu_k \in M$ which satisfies $\gamma$-margin and $\mu_i$ is the mean of points in $C_i$. Given $\epsilon, \delta \in (0, 1)$. If $\gamma \ge 1 + 2\epsilon$ then with probability atleast $1-\delta$, Alg. \ref{alg:steinerQueryPositive} outputs a clustering $\mc C$ such that $\mc C = \mc C_{\mc X}$.
\end{theorem}

\begin{proof}
Using Lemma \ref{lemma:phase1}, we get that with probability atleast $1-\delta/k$, phase one chooses a center $\mu_p'$ such that $d(\mu_p, \mu_p') \le r(C_p)\epsilon$. Lemma \ref{lemma:hasGammaMargin} then implies that $d(x, \mu_p') < d(y, \mu_p')$ for all $x \in C_p$ and $y \not\in C_p$. Hence, $b_{idx}$ in phase two of Alg. \ref{alg:steinerQueryPositive} is such that $b_{idx} = \argmax\limits_{x \in C_p} d(x, \mu_p')$. This implies that $C_p'$ (found in phase two) equals $C_p$. Hence, with probability atleast $1-\delta/k$ one iteration of the algorithm successfully finds all the points in a cluster $C_p$. Using union bound, we get that with probability atleast $1-k\delta/k = 1-\delta$, the algorithm recovers the target clustering.
\qed 
\end{proof}

\begin{theorem}
\label{thm:steinerQueryPositiveComplexity}
Let the framework be as in Thm. \ref{thm:steinerQueryPositive}. Then Alg. \ref{alg:steinerQueryPositive} 
\begin{itemize}[nolistsep,noitemsep]
\item makes $O\big(k\log |\mc X| + k^2\frac{\log k + \log (1/\delta)}{\epsilon^4}\big)$ queries to the oracle $\mc O$.
\item runs in $O\big(k|\mc X|\log |\mc X| + k^2\frac{\log k + \log (1/\delta)}{\epsilon^4}\big)$ time.
\end{itemize}
\end{theorem}

\begin{proof}
The first phase of the algorithm takes $O(k\eta)$ time and makes $k\eta+1$ queries to the oracle. The second phase takes $O(n\log n)$ times and makes $O(\log n)$ queries to the oracle. Hence, in total the algorithm takes $O(kn\log n + k^2\eta)$ time and makes $O(k\log n + k^2\eta)$ queries. Substituting $\eta = c\frac{\log k + \log(1/\delta)}{\epsilon^4}$ gives the desired result.
\qed
\end{proof}

\section{Lower Bounds}
\label{section:lowerBounds}
In section \ref{section:clusteringWithQuery}, we proposed an algorithm which for $\gamma > 1$, makes $O(k\log n)$ queries and in polynomial time finds the target clustering with very high probability. In this section, we want to give a lower bound for our problem setting. We will show that if $\gamma < 1.4$ then finding the optimal euclidean $k$-means clustering is NP-Hard without queries. Note that Thms. \ref{thm:steinerQueryPositive} and \ref{thm:steinerQueryPositiveComplexity} shows that we can overcome this lower bound using queries. Next, we also give a lower bound on the number of queries needed. We show that making `too few' (sublogarithmic in $k$) queries also doesn't help and the problem remains NP-Hard.

\subsection{Lower bounds without query}
Given a clustering instance $(\mc X, d)$ in euclidean space and the number of clusters $k$. The $k$-means tries to find a clustering $\mc C = \{C_1, \ldots, C_k\}$ which minimizes the following objective function. $\sum\limits_{C_i}\frac{1}{|C_i|} \sum_{x, y \in C_i} d^2(x, y)$. \cite{vattani2009hardness} considered the weighted version of this problem where every point $x$ has a weight $w(x)$. $$\sum\limits_{C_i}\frac{1}{\sum_{x \in C_i}w(x)} \sum_{x, y \in {C_i \choose 2}} w(x)w(y)d^2(x, y).$$ Vattani showed that finding the optimal solution to the weighted $k$-means objective is NP-Hard even in Euclidean plane. We will use the same reduction technique and ideas as Vattani's proof to get our lower bound. However to prove the lower bound under gamma margin, we need to modify the author's original proof and argue more carefully.

\begin{theorem}
For any $\gamma < 1.4$, finding the optimal solution to the $k$-means objective function is NP-Hard even when the optimal satisfies $\gamma$-margin.
\end{theorem}
To prove the theorem, we will reduce an instance of exact cover by 3-sets (X3C) to the decision version of weighted $k$-means (which asks for a clustering with cost $\le L$).
\begin{definition}[X3C]
Given a set $U$ containing exactly $3n$ elements and a collection $\mc S = \{S_1, \ldots, S_l\}$ of subsets of $X$ such that each $S_i$ contains exactly three elements. Does there exist $n$ elements in $\mc S$ such that their union is $U$? 
\end{definition}

\subsubsection{Component design}
We will now describe the reduction from X3C to the $k$-means problem. We use the same reduction that was used by \cite{vattani2009hardness}. 

The component $H_{l,n}$ and its parameters are described in Fig. \ref{fig:lowerBoundComponent}. The row $R_i$ is composed of $6n + 3$ points $\{s_i, r_{i, 1}, \ldots, r_{i, 6n+1}, f_i\}$. The points $r_{i, j}$ have weight $w$ and points $s_i$ and $f_i$ have weight $2w$. Row $M_i$ is composed of $3n$ points $\{m_{i,1}, \ldots, m_{i, 3n}\}$ of weight $2w$. The distances between the points are shown in Fig. \ref{fig:lowerBoundComponent}. We choose $k = (l-1)3n + l(3n+2)$. 

We choose the following values of the parameters. $h = \sqrt{2}, d = \sqrt{4.5}, \epsilon = \frac{1}{w^2}$ and $\alpha = \frac{d}{w}-\frac{1}{2w^3}.$

\begin{definition}[$A$ and $B$ clustering of $R_i$ \cite{vattani2009hardness}]

\noindent $A$ - For $1 \le j \le 3n$, $R_i$ has clusters $\{r_{2j-1}, r_{2j}\}$. Also, it has clusters $\{s_i\}, \{r_{i, 6n+1}, f_i\}$\\
\noindent $B$ - For $1 \le j \le 3n$, $R_i$ has clusters $\{r_{2j}, r_{2j+1}\}$. Also, it has clusters $\{s_i, r_{i, 1}\}, \{f_i\}$
\end{definition}
A row grouped in a $A$ clustering costs $(6n+3)w-\alpha$ while a row in $B$ clustering costs $(6n+3)w$. Hence, the clustering with the rows $R_i$ grouped in a $A$ or $B$ clustering and the rows $M_i$ in single clusters is a $k$-clustering and costs atmost $L_1 := (6n+3)lw$. We will call such a clustering `{\it nice}'.

\begin{lemma}[Similar to Lemma 7 in \cite{vattani2009hardness}]
Any non-nice clustering of $H_{l, n}$ costs atleast $> L_1 + \frac{7}{8}w$ where $w = poly(l, n)$
\end{lemma}

\begin{proof}
Any non-nice clustering $C$ can be of the following different forms.
\begin{itemize}[nolistsep]
\item Contains $m_{i, j}, m_{i,j+1}$ in the same cluster - $C$ must have either of the following. (1) $r_{i, j}$ and $r_{i,j+1}$ as a singleton. In this case, the new cost is atleast $L_1 + 8w - 2w = L_1+6w$. (2) $s_i$ and $r_{i,1}$ as singleton. In this case, the difference of cost is atleast $8w - 3w = 5w$. (3) $f_i$ and $r_{i,6n+1}$ as singleton. In this case, the difference of cost is atleast $8w - (3w-\alpha) = 5w+\alpha$. Note that having multiple $m_{i, j}$'s in the same cluster or having multiple clusters which contain $m_{i, j}, m_{i,j+1}$ will bring the cost up even furthur.
\item Contains $r_{i, 2j}, m_{i, j}$ in the same cluster - $C$ must have either of the following. (1) $r_{i, j'}$ as a singleton. In this case, the cost difference is atleast $ \frac{2w}{3}(\sqrt2+1)^2 - 2w > \frac{15}{8}w$. (2) $s_i$ and $r_{i,1}$ as singleton. In this case, the difference of cost is atleast $\frac{2w}{3}(\sqrt2+1)^2 - 3w > \frac{7}{8}w$. (3) $f_i$ and $r_{i,6n+1}$ as singleton. In this case, the difference of cost is atleast $\frac{2w}{3}(\sqrt2+1)^2 - (3w-\alpha) > \frac{7}{8}w+\alpha$. Note that having multiple clusters which contain $m_{i, j}, r_{i',j'}$ will bring the cost up even furthur.
\item Contains $r_{i,j}$ as singleton - $C$ must contain both $\{s_i, r_{i,1}\}$ and $\{r_{i,6n+1}, f_i\}$. Hence the increase in cost is atleast $3w - 2w = w$.
\item Contains multiple $\{r_{i,j}\}$ in the same cluster - Assume the cluster has cardinality $m > 2$. Using simple counting argument (as has also been shown in  \cite{vattani2009hardness}), this cluster costs atleast $\frac{w}{3}m(m^2-1)$. A nice clustering would cost atmost $w(m + \lceil{\frac{m}{2}}\rceil - 2) + 3w = w(m + \lceil{\frac{m}{2}}\rceil + 1)$. Hence, the difference in cost is atleast $2w$.
\end{itemize}
\qed
\end{proof}

\subsubsection{Reduction and proof}
Given an instance of X3C, that is, given elements $U = \{1, \ldots, 3n\}$ and collection $\mc S$, we construct an instance $X$ of $k$-means as follows. $X = H_{l,n} \cup (\cup_{i=1}^l Z_i)$ where $Z_i$ depends on the set $S_i$. 

The set $Z_i$ is shown in Fig. \ref{fig:setZDescription}. The construction is same as \cite{vattani2009hardness}. For every $j\in U$, there are four possible locations $x_{i, j}, x_{i,j'}, y_{i,j}$ and $y_{i, j}'$. One of $x_{i,j}$ and $x_{i,j}'$ will be occupied and one of $y_{i,j}$ and $y_{i,j}'$ will be occupied. Each point has weight $\frac{w}{2}$.
\begin{itemize}[nolistsep,noitemsep]
\item $j \in S_i \iff x_{i,j}' \in Z_i$
\item $j \not\in S_i \iff x_{i,j} \in Z_i$
\item $j \in S_{i+1} \iff y_{i,j}' \in Z_i$
\item $j \not\in S_{i+1} \iff y_{i,j} \in Z_i$
\end{itemize}

Hence, our clustering instance $X = H_{l,n} \cup Z$. The number of clusters $k = (l-1)3n + l(3n+2)$ and the cost $L = L_1 + L_2 -n\alpha$. We choose $L_2 = 6n(l-1)\frac{2w\frac{w}{2}}{2w+\frac{w}{2}}h^2 = 6n(l-1)$.


Use a variant of Vattani proof.

\subsection{Lower bound with sub-logarithmic queries}
Proof as simulation.


\section{Discussion and conclusion}
Other related works and conclusion.

\bibliographystyle{alpha}
\bibliography{activeClustering}

\appendix
\section{Relationship between two different query models}
\label{appendix:diffQueryModels}

\section{Comparison of $\gamma$-margin and $\alpha$-center proximity}
\label{appendix:gammaMrginVsAlphaCenter}


\section{Center Proximity}
We are given a clustering instance $(\mc X, d)$ in some metric space $M$. A center-based clustering is induced by centers $c_1, \ldots, c_k \in M$ where each point $x \in \mc X$ is assigned to its closest center.

\begin{definition}[$\alpha$-center proximity]
\label{defn:alphacp}
Given a clustering instance $(\mc X, d)$ and the number of clusters $k$. We say that a clustering $\mc C_{\mc X} = \{C_1, \ldots, C_k\}$ induced by centers $c_1, \ldots, c_k \in M$ has $\alpha$-center proximity w.r.t $\mc X$ and $k$ if the following holds. For all $x \in C_i$ and $i\neq j$, 
$$\alpha d(x, c_i) < d(x, c_j)$$
\end{definition}

\subsection{Goal}
We are given a dataset $\mc X$. $\mc X$ has a target clustering $\mc C_{\mc X}$ which satisfies $\alpha$-center proximity. Our goal is to recover the target clustering $\mc C_{\mc X}$. There is a catch. Our algorithm has access to an {\it oracle} which can provide answers to membership queries. That is, we can pose the following question to the oracle.
\begin{center}
  \begin{tabular}{l}
	{\it Question:} Does $x \in \mc X$ belong to the $i^{th}$ cluster $C_i$ ? \\
	{\it Oracle:} Answers `yes' or `no'
  \end{tabular}
\end{center}

\noindent Our goal is to design an algorithm which given a set $\mc X$ and an oracle $\mc O$ outputs the target clustering $\mc C_{\mc X}$ while making as few queries to the oracle as possible. Note that it is always possible recover the target clustering by making $n = |\mc X|$ queries to the oracle. Hence, we will require that any algorithm make sub-linear queries to the oracle. 

\subsubsection*{Problem Setting}
\begin{itemize}[nolistsep, noitemsep]
\item {\it Steiner} - In this setting, centers $c_1, \ldots, c_k$ can be arbitrary points in the metric space.
\item {\it Restricted} - The centers are part of the data-set $\mc X$, that is, $c_1, \ldots, c_k \in \mc X$ \\
\end{itemize}
We will give algorithms in both the settings. Note that restricted setting is similar to working with $k$-median cost function (if we consider the target clustering to be the optimal $k$-median clustering). Similarly, the optimal $k$-means solution can be thought of as a target in the steiner setting.

\subsection{Related work}
\label{section:relatedwork}
For $\alpha$-center proximity, the following results are known when no oracle (or queries) is available to the clustering algorithm. In the restricted setting, the algorithm of Balcan and Liang \cite{balcan2012clustering} `recovers' the target clustering (outputs a tree such that the target is a pruning of the tree) when $\alpha > \sqrt{2} + 1$. In this work, we will show that in the presence of queries, we can recover the target clustering as long as $\alpha > 2$. Note that this is still above the NP-Hardness lower bound of $\alpha < 2$ in the restricted setting (\cite{ben2014data}).

In the steiner setting, the algorithm of Awasthi et. al \cite{awasthi2012center} recovers the target clustering when $\alpha > 2+\sqrt{3}$. We show that in the presence of few queries, we can recover the target clustering if $\alpha > \sqrt{2}+1$. This is much better than the previous known result in the absence of queries. This result also beats the NP-Hardness lower bound of $\alpha < 3$ in the steiner setting (\cite{awasthi2012center}).

\subsection{Algorithm}


\subsection{Lower bound}
In the steiner setting, Awasthi et. al \cite{awasthi2012center} showerd that for $\alpha < 3$ it is NP-Hard to find the optimal solution to the $k$-median objective function. Our positive result (Thm. \ref{thm:steinerQueryPositive}) is for the $k$-means optimal clustering. Hence, we would like to prove lower bounds in this setting. We will show that for $\alpha < 3$, it is NP-Hard to approximate the $k$-means cost function as well. The proof uses a simple reduction from the $k$-median instance to the $k$-means instance and is stated below.

\begin{theorem}
For any $\alpha < 3$, finding the optimal solution to the $k$-means objective function is even when the optimal solution satisfies $\alpha$-center proximity is NP-hard.
\end{theorem}

\begin{proof}
We will first state both the $k$-median and the $k$-means optimization problem under $\alpha$-center proximity and then show the reduction.

\vspace{1em}\noindent $k$-media clustering\\
{\it Input:} A set $\mc X$, integer $k$ and metric $d$. It is known that the desired solution satisfies $\alpha$-center proximity.\\
{\it Output:} A partition of $\mc X \subseteq M$ into $k$ clusters $C_1, \ldots, C_k$ with a center $c_i \in M$ for each cluster so as to minimize $\sum_{C_i}\sum_{x \in C_i} d(x, c_i)$.  

\vspace{0.5em}\noindent $k$-means clustering\\
{\it Input:} A set $\mc X$, integer $k$ and metric $d$. It is known that the desired solution satisfies $\alpha$-center proximity.\\
{\it Output:} A partition of $\mc X \subseteq M$ into $k$ clusters $C_1, \ldots, C_k$ with a center $c_i \in M$ for each cluster so as to minimize $\sum_{C_i}\sum_{x \in C_i} d^2(x, c_i)$.  

\vspace{1em}\noindent Awasthi et. al \cite{awasthi2012center} showed that the $k$-median problem above is NP-Hard. Given an instance of $k$-median problem we construct an instance of $k$-means as follows. We, let $\mc X, k$ be the same. But we use a new metric $\tilde d = \frac{d}{\sqrt{|\mc X|}}$. We will show that if the $k$-median instance has cost $\le A$ if and only if $k$-means instance has cost $\le A^2$. In the proof, we use $c(x)$ to denote the center of $x$. This will prove our desired result.

\noindent$\Rightarrow$ For the forward direction, observe that $\sum_x \tilde d^{2}(x, c_i) \le\sum_{x}d^2(x, c(x)) \le (\sum_{x} d(x, c_i))^2 \le A^2$.

\noindent$\Leftarrow$ For the reverse direction, observe that $(\sum_{x}d(x, c(x)))^2 \le n\sum_{x} d^2(x, c_i) = \sum_x \tilde d^{2}(x, c_i) \le A^2$.
\end{proof}








\section{Concentration inequalities}
\label{appendixsection:conIneq}

\begin{theorem}[Generalized Hoeffding's Inequality \cite{ashtiani2015dimension}]
\label{thm:genHoeff}
Let $X_1, \ldots. X_n$ be i.i.d random vectors in some Hilbert space such that for all $i$, $\|X_i\|_2 \le R$ and $E[X_i] = \mu$. If $n > c\frac{\log(1/\delta)}{\epsilon^2}$, then with probability atleast $1-\delta$, we have that
$$\Big\|\mu - \frac{1}{n}\sum X_i\Big\|_2^2 \le R^2\epsilon$$ 
\end{theorem}

\section{Some properties of Center Proximity (maybe useful later)}
\begin{lemma}
\label{lemma:hasPropertyR}
Given a clustering instance $(\mc X, d)$ in some metric space $M$. Let $\mc C_{\mc X} = \{C_1, \ldots, C_k\}$ be a clustering of $\mc X$ induced by centers $c_1, \ldots, c_k \in M$ which satisfies $\alpha$-center proximity. Let $c_i' \in M$ such that $d(c_i, c_i') \le r(C_i)\epsilon$. If $\alpha \ge 2 + 3\epsilon$ then for all $x \in C_i$ and for all $y \in C_j$
$$d(x, c_i') < d(x, y)$$  
\end{lemma}

\begin{proof}
Fix any $x \in C_i$ and $y \in C_j$. Let $c_i, c_i'$ and $c_j$ be as defined in the statement of the lemma. Let $x^*$ be such that $d(x^*, c_i) = r(C_i)$. The proof of our lemma will follow from the following two properties of $\alpha$-center proximity instances.
\begin{enumerate}[nolistsep,noitemsep]
\item $r(C_i) < \frac{1}{\alpha-1}d(c_i, c_j)$.
\begin{flalign*}
&(\alpha-1) r(C_i) = \alpha d(x^*, c_i) - d(x^*, c_i) < d(x^*, c_j) - d(x^*, c_i) \le d(c_i, c_j)&
\end{flalign*}
\item $d(c_i, c_j) < \frac{\alpha+1}{\alpha-1}d(x, y)$
\begin{flalign*}
&\text{Using the triangle inequality, we get that}&\\
&d(c_i, c_j) \le d(x, y) + d(x, c_i) + d(y, c_j). \text{From Corollary $2.3$ in \cite{awasthi2012center}, we know that}&\\
&d(x, c_i), d(y, c_j)< d(x, y)/(\alpha-1) \text{ which completes the proof of the result.}&
\end{flalign*}
\end{enumerate}
\begin{flalign*}
&\text{Now to prove the lemma, observe that using triangle inequality, we get that
}&\\
&d(x, c_i') \le d(c_i, c_i') + d(x, c_i) < \frac{d(x, y)}{(\alpha-1)} + \epsilon r(C_i) < \bigg[ \frac{1}{\alpha-1} + \frac{\epsilon(\alpha+1)}{(\alpha-1)^2}\bigg]d(x, y) = \frac{(1+\epsilon)\alpha-1+\epsilon}{(\alpha-1)^2}d(x, y)&
\end{flalign*}
Observe that, $(\alpha-1)^2 \ge (1+\epsilon)\alpha -1 + \epsilon \iff \alpha^2 -(3+\epsilon)\alpha + 2-\epsilon \ge 0$
$\iff \alpha \ge \frac{3+\epsilon + \sqrt{1+10\epsilon + \epsilon^2}}{2}$ or $\alpha \le \frac{3+\epsilon - \sqrt{1+10\epsilon + \epsilon^2}}{2}$. Now, for $\alpha \ge 2 + 3\epsilon \implies \alpha \ge \frac{3+\epsilon + \sqrt{(1+5\epsilon)^2}}{2}$ which completes the proof of the lemma.
\end{proof}

\end{document}


